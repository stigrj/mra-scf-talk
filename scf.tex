
\begin{frame}
    \centering
    \textbf{\Large{Iterative solution algorithms}}
\end{frame}

\begin{frame}
    \frametitle{One-electron systems}
    \centering
    \textbf{Potential operator}
    \begin{equation}
	\nonumber
	\hat{V} = v_{nuc}(r) = -\sum_I\frac{Z_I}{|r-R_I|}
    \end{equation}

    \vspace{5mm}

    \textbf{Smoothed nuclear potential}
    \begin{align}
	\nonumber
	\frac{1}{r} &\approx \frac{erf(r)}{r} +
	\frac{1}{3\sqrt{\pi}}\big(e^{-r^2}+16e^{-4r^2}\big)
    \end{align}

    \vspace{5mm}

    \textbf{One-electron Schr\"{o}dinger equation}
    \begin{equation}
        \nonumber
        \Big[-\frac{1}{2}\nabla^2 + \hat{V}\Big]\psi(r) = E \psi(r)
    \end{equation}

    \vspace{1mm}

    \begin{equation}
        \nonumber
        \psi = -2\hat{G} \Big[\hat{V} \psi \Big]
    \end{equation}

    \vspace{5mm}

    \centering
    \tiny
    RJ Harrison, GI Fann, T Yanai, Z Gan and G Beylkin,
    {\it J. Chem. Phys.}, 
    \textbf{121},
    11587 (2004)
\end{frame}

\begin{frame}
    \frametitle{One-electron algorithm}
    \centering
    \textbf{Power iteration of the BSH operator}
    \begin{equation}
	\nonumber
	\psi^{n+1} = -2G^n\Big[\hat{V} \psi^n\Big]
    \end{equation}

    \vspace{3mm}

    \textbf{Finding roots of the residual}
    \begin{equation}
	\nonumber
	f(\psi) = -2G^n\big[\hat{V} \psi\big] -\psi
    \end{equation}

    \vspace{3mm}

    \textbf{Newton's method}
    \begin{equation}
	\nonumber
	\psi^{n+1} = \psi^n - \Big[J(\psi^n)\Big]^{-1} f(\psi^n)
    \end{equation}

    \begin{equation}
	\nonumber
	\psi^{n+1} = \psi^n - \Big[J(\psi^n)\Big]^{-1}
	\bigg(-2G^n\Big[\hat{V}\psi^n\Big] - \psi^n\bigg)
    \end{equation}

    \vspace{3mm}

    So the direct power iteration is an "inexact" Newton method\\
    where we approximate the Jacobian $J(\psi^n) \approx -1$.
\end{frame}

\begin{frame}
    \frametitle{One-electron algorithm}
    \begin{columns}
    \begin{column}{.50\textwidth}
    \centering
    \textbf{Initialize BSH operator} $\hat{G}^n$
    \begin{equation}
        \nonumber
        \mu^n = \sqrt{-2E^n}
    \end{equation}
    \end{column}

    \begin{column}{.50\textwidth}
    \centering
    \textbf{Power iteration}
    \begin{equation}
	\nonumber
	\tilde{\psi}^{n+1} = -2\hat{G}^n \Big[ \hat{V} \psi^n \Big]
    \end{equation}
    \end{column}
    \end{columns}

    \vspace{5mm}

    \begin{columns}
    \begin{column}{.50\textwidth}
    \centering
    \textbf{Wavefunction update}
    \begin{equation}
	\nonumber
	\Delta\psi^n = \frac{\tilde{\psi}^{n+1}}{\|\tilde{\psi}^{n+1}\|} - \psi^n
    \end{equation}
    \end{column}

    \begin{column}{.50\textwidth}
    \centering
    \textbf{Energy update}
    \begin{equation}
	\nonumber
	\Delta E^n =
        \frac{\langle\tilde{\psi}^{n+1}|\hat{V}|\Delta\tilde{\psi}^n\rangle}
        {\langle\tilde{\psi}^{n+1}|\tilde{\psi}^{n+1}\rangle}
    \end{equation}
    \end{column}
    \end{columns}

    \vspace{10mm}

    \centering
    \textbf{Update wavefunction and energy}
    \begin{align}
	\nonumber
        \psi^{n+1}  &= \psi^n + \Delta \psi^n\\
	\nonumber
        E^{n+1}     &= E^n + \Delta E^n
    \end{align}
\end{frame}

\begin{frame}
    \frametitle{Energy update}
    \centering
    \textbf{Using the relation}
    \begin{equation}
        \nonumber
        -2\hat{G}^n = \big(\hat{T} - E^n\big)^{-1}
    \end{equation}

    \vspace{5mm}

    \textbf{Manipulating the energy expression}
    \begin{align}
        \tilde{E}^{n+1}
        \nonumber
        &=	\langle\tilde{\psi}^{n+1}| \hat{T}+\hat{V} | \tilde{\psi}^{n+1}\rangle\\
        \nonumber
        &=	\langle\tilde{\psi}^{n+1}|  \hat{T} - E^n  | \tilde{\psi}^{n+1}\rangle
        +	\langle\tilde{\psi}^{n+1}|  E^n + \hat{V}  | \tilde{\psi}^{n+1}\rangle\\
        \nonumber
        &=	\langle\tilde{\psi}^{n+1}|  \hat{T} - E^n  | 
	        -2\hat{G}^n\big[\hat{V}\psi^n\big]\rangle
        +	\langle\tilde{\psi}^{n+1}| E^n + \hat{V} |\tilde{\psi}^{n+1}\rangle\\
        \nonumber
        &= -\langle\tilde{\psi}^{n+1}| \hat{V} |\psi^{n}\rangle
        +	\langle\tilde{\psi}^{n+1}| E^n + \hat{V} |\tilde{\psi}^{n+1}\rangle\\
        \nonumber
        &= E^{n}\langle\tilde{\psi}^{n+1}|\tilde{\psi}^{n+1}\rangle + 
	    \langle\tilde{\psi}^{n+1}| \hat{V} |\Delta\tilde{\psi}^{n}\rangle
    \end{align}

    \vspace{8mm}

    \centering
    \textbf{Energy without kinetic operator}
    \begin{equation}
        \nonumber
        E^{n+1} = E^{n} + 
        \frac{\langle\tilde{\psi}^{n+1}| \hat{V} |\Delta\tilde{\psi}^{n}\rangle}
        {\langle\tilde{\psi}^{n+1}|\tilde{\psi}^{n+1}\rangle}
    \end{equation}
\end{frame}

\begin{frame}
    \frametitle{Hydrogen atom}
    \begin{center}
	\includegraphics[scale=1.0, clip, viewport = 50 550 300 730]{figures/h_convergence.pdf}
    \end{center}
\end{frame}

\begin{frame}
    \frametitle{Krylov subspace Accelerated Inexact Newton (KAIN)}
    \begin{columns}
    \begin{column}[b]{0.5\textwidth}
    \centering
    \textbf{Wavefunction history}
    \begin{equation}
	\nonumber
	\psi^0, \psi^1, \dots, \psi^n
    \end{equation}
    \end{column}
    \begin{column}[b]{0.5\textwidth}
    \centering
    \textbf{Residual history}
    \begin{equation}
	\nonumber
	f(\psi^0), f(\psi^1), \dots, f(\psi^n)
    \end{equation}
    \end{column}
    \end{columns}

    \vspace{5mm}

    \centering
    \textbf{Used to find a better approximation to the Jacobian}

    \vspace{10mm}

    The new iterative step is then expanded in the Krylov subspace
    \begin{equation}
	\nonumber
	\delta\psi^n = \sum_i c_i\Big(\psi^i-\psi^n\Big) - 
	\sum_i c_i\Big(f(\psi^i) - f(\psi^n)\Big) - f(\psi^n)
    \end{equation}

    \vspace{5mm}

    by solving the linear system $Ac = b$
    \begin{align}
	\nonumber
	A_{ij} &= \langle\psi^n-\psi^i|f(\psi^n) - f(\psi^j)\rangle\\
	\nonumber
	b_{i}  &= \langle\psi^n-\psi^i|f(\psi^n)\rangle
    \end{align}

    \vspace{5mm}

    \centering
    \tiny
    RJ Harrison,
    {\it J. Comput. Chem.}, 
    \textbf{25(3)},
    328 (2004)
\end{frame}

\begin{frame}
    \frametitle{One-electron algorithm}

    \begin{columns}
    \begin{column}{.50\textwidth}
    \centering
    \textbf{Initialize BSH operator} $\hat{G}^n$
    \begin{equation}
        \nonumber
        \mu^n = \sqrt{-2E^n}
    \end{equation}
    \end{column}

    \begin{column}{.50\textwidth}
    \centering
    \textbf{Power iteration}
    \begin{equation}
	\nonumber
	\tilde{\psi}^{n+1} = -2\hat{G}^n \Big[ \hat{V} \psi^n \Big]
    \end{equation}
    \end{column}
    \end{columns}

    \vspace{5mm}

    \begin{columns}
    \begin{column}{.50\textwidth}
    \centering
    \textbf{Wavefunction update}
    \begin{equation}
	\nonumber
	\Delta\psi^n = \frac{\tilde{\psi}^{n+1}}{\|\tilde{\psi}^{n+1}\|} - \psi^n
    \end{equation}
    \end{column}

    \begin{column}{.50\textwidth}
    \centering
    \textbf{Energy update}
    \begin{equation}
	\nonumber
	\Delta E^n =
        \frac{\langle\tilde{\psi}^{n+1}|\hat{V}|\Delta\tilde{\psi}^n\rangle}
        {\langle\tilde{\psi}^{n+1}|\tilde{\psi}^{n+1}\rangle}
    \end{equation}
    \end{column}
    \end{columns}

    \vspace{5mm}

    \centering
    \textbf{KAIN update}
    \begin{equation}
	\nonumber
        \left(
        \begin{matrix}
        \delta \psi\\
        \delta E
        \end{matrix}
        \right)^{n+1}
        \longleftarrow
        \left(
        \begin{matrix}
        \psi\\
        E
        \end{matrix}
        \right)^n
        ,
        \left(
        \begin{matrix}
        \Delta \psi\\
        \Delta E
        \end{matrix}
        \right)^n
    \end{equation}

    \vspace{5mm}

    \textbf{Update wavefunction and energy}
    \begin{align}
	\nonumber
        \psi^{n+1}  &= \psi^n + \delta \psi^n\\
	\nonumber
        E^{n+1}     &= E^n + \delta E^n
    \end{align}
\end{frame}

\begin{frame}
    \frametitle{Hydrogen atom}
    \begin{center}
	\includegraphics[scale=1.0, clip, viewport = 300 550 560 740]{figures/h_convergence.pdf}
    \end{center}
\end{frame}

\begin{frame}
    \frametitle{Many-electron systems}
    \centering
    \textbf{Potential operator} $0 \leq \alpha \leq 1$
    \begin{equation}
        \nonumber
        \hat{V} = v_{nuc}(r) + v_{el}(r) + v_{xc}(r) - \alpha\hat{K}
    \end{equation}

    \vspace{10mm}

    \begin{columns}
    \begin{column}[b]{0.5\textwidth}
    \centering
    \textbf{Classical nuclear}
    \begin{equation}
        \nonumber
	v_{nuc}(r) = -\sum_I\frac{Z_I}{|r-R_I|}
    \end{equation}
    \end{column}

    \begin{column}[b]{0.5\textwidth}
    \centering
    \textbf{Classical Coulomb}
    \begin{equation}
        \nonumber
        v_{el}(r) = \int \frac{\rho(r')}{4\pi|r-r'|} \ud r'
    \end{equation}
    \end{column}
    \end{columns}

    \vspace{5mm}

    \begin{columns}
    \begin{column}[b]{0.5\textwidth}
    \centering
    \textbf{Exchange-Correlation}
    \begin{equation}
        \nonumber
        v_{xc}(r)
        %= \frac{\delta E_{xc}}{\delta \rho}
        = \frac{\partial F_{xc}}{\partial \rho} - \nabla \cdot \frac{\partial F_{xc}}{\partial \nabla\rho}
    \end{equation}
    \end{column}

    \begin{column}[b]{0.5\textwidth}
    \centering
    \textbf{Hartree-Fock exchange}
    \begin{equation}
        \nonumber
        \hat{K}\phi_p(r) = \sum_i \phi_i(r) \int \frac{\phi_i^\dagger(r')\phi_p(r')}{4\pi|r-r'|} \ud r'
    \end{equation}
    \end{column}
    \end{columns}
\end{frame}

\begin{frame}
    \frametitle{Orthonormalization}
    \centering
    \textbf{Straightforward iteration of the Kohn-Sham equations}
    \begin{equation}
        \nonumber
        \tilde{\phi}_i^{n+1} = -2\hat{G}_i^n \bigg[\hat{V}^n\phi_i^n\bigg]
    \end{equation}
    \textbf{brings all orbitals to the lowest energy eigenfunction}
    
    \vspace{15mm}

    \textbf{Orthonormality must be imposed}
    \begin{equation}
        \nonumber
        \tilde{S}_{ij} = \langle\tilde{\phi}_i|\tilde{\phi}_j\rangle = \delta_{ij}
    \end{equation}

    \vspace{5mm}

    \begin{columns}
    \begin{column}[b]{0.5\linewidth}
    \centering
    \textbf{Gram-Schmidt}
    \begin{equation}
	\nonumber
	\phi_i = \Big(1 - \sum_{j<i}|\phi_j\rangle\langle\phi_j|\Big)\tilde{\phi}_i
    \end{equation}
    \end{column}

    \begin{column}[b]{0.5\linewidth}
    \centering
    \textbf{L\"{o}wdin orthonormalization}
    \begin{equation}
	\nonumber
	\phi_i = \sum_j \tilde{S}_{ij}^{-1/2}\tilde{\phi}_j
    \end{equation}
    \end{column}
    \end{columns}
\end{frame}

\begin{frame}
    \frametitle{Non-canonical orbitals}
    \centering
    \textbf{The non-canonical Kohn-Sham equations}
    \begin{equation}
        \nonumber
        \hat{F}|\phi_i\rangle 
        = \bigg[\sum_j|\phi_j\rangle\langle\phi_j|\bigg]\hat{F}|\phi_i\rangle
        = \sum_jF_{ji}|\phi_j\rangle
    \end{equation}

    \vspace{5mm}

    \textbf{Rewrite using} $\mu_i^2 = -2\lambda_i$
    \begin{align}
        \nonumber
        \bigg[-\frac{1}{2}\nabla^2 + \hat{V}\bigg]\phi_i
        &= \sum_jF_{ji}\phi_j\\
        \nonumber
        \bigg[-\nabla^2 + \mu_i^2\bigg]\phi_i 
        &= -2\bigg[\hat{V}\phi_i - \sum_j\big(F_{ji} -
        \lambda_i\delta_{ji}\big)\phi_j\bigg]\\
        \nonumber
        \phi_i
        &= -2\hat{G}_i\bigg[\hat{V}\phi_i - \sum_j\big(F_{ji} -
        \Lambda_{ji}\big)\phi_j\bigg]
    \end{align}

    \vspace{5mm}

    \begin{columns}
    \begin{column}[b]{0.48\linewidth}
    \centering
    \textbf{Fock matrix}
    \begin{equation}
        \nonumber
        F_{ij} = \langle\phi_i|\hat{T} + \hat{V}|\phi_j\rangle
    \end{equation}
    \end{column}

    \begin{column}[b]{0.48\linewidth}
    \centering
    \textbf{Kinetic matrix}
    \begin{equation}
        \nonumber
        T_{ij}
        = \langle\phi_i|\hat{T}|\phi_j\rangle
        = \frac{1}{2}\langle\nabla\phi_i|\nabla\phi_j\rangle
    \end{equation}
    \end{column}
    \end{columns}
\end{frame}

\begin{frame}
    \frametitle{Localized orbitals}
    \centering
    Total energy invariant under unitary transformations among occupied orbitals
    \begin{equation}
	\nonumber
	\phi_i = \sum_j L_{ij} \phi_i, \qquad \qquad L^\ast L = LL^\ast = I
    \end{equation}

    \begin{columns}
    \begin{column}[b]{0.48\linewidth}
    \centering
    \includegraphics[scale=0.25, clip, viewport = 80 560 600 700]{figures/alkane.pdf}\\
    \includegraphics[scale=0.25, clip, viewport = 80 560 600 700]{figures/can_orb_1.pdf}\\
    \includegraphics[scale=0.25, clip, viewport = 80 560 600 700]{figures/can_orb_2.pdf}\\

    \vspace{2mm}

    \begin{equation}
        \nonumber
        \phi_i =\ -2\hat{G}_i\Bigg[\hat{V}\phi_i
        - (\epsilon_i - \lambda_i)\phi_i\Bigg]
    \end{equation}
    \end{column}

    \begin{column}[b]{0.48\linewidth}
\only<2>{
    \centering
    \includegraphics[scale=0.25, clip, viewport = 80 560 600 700]{figures/loc_orb_1.pdf}\\
    \includegraphics[scale=0.25, clip, viewport = 80 560 600 700]{figures/loc_orb_2.pdf}\\
    \includegraphics[scale=0.25, clip, viewport = 80 560 600 700]{figures/loc_orb_3.pdf}\\

    \vspace{2mm}

    \begin{equation}
        \nonumber
        \phi_i =\ -2\hat{G}_i\Bigg[\hat{V}\phi_i
        - \sum_j(F_{ji} - \Lambda_{ji})\phi_j\Bigg]
    \end{equation}
}
    \end{column}
    \end{columns}

    \vspace{6mm}

    \centering
    \tiny
    S.F. Boys,
    {\it Rev. Mod. Phys.}, 
    \textbf{32:296}
    (1960)\\
    J.M. Foster, S.F. Boys,
    {\it Rev. Mod. Phys.}, 
    \textbf{32:300}
    (1960)
\end{frame}

%\begin{frame}
%    \frametitle{Orthonormalization}
%    \centering
%    \textbf{The non-canonical Kohn-Sham equations}
%    \begin{equation}
%        \nonumber
%        \tilde{\phi}_i^{n+1} = -2\hat{G}_i^n \bigg[\hat{V}^n\phi_i^n -
%        \sum_j\big(F_{ji}^n - \Lambda_{ji}^n\big)\phi_j^n\bigg]
%    \end{equation}
%    
%    \vspace{10mm}
%
%    \textbf{Overlap matrix}
%    \begin{equation}
%        \nonumber
%        \tilde{S}_{ij} = \langle\tilde{\phi}_i|\tilde{\phi}_j\rangle
%    \end{equation}
%
%    \vspace{10mm}
%
%    \begin{columns}
%    \begin{column}[b]{0.3\linewidth}
%    \centering
%    \textbf{Diagonalize}
%    \begin{equation}
%	\nonumber
%        U = D\tilde{S}^{-1/2}
%    \end{equation}
%    \end{column}
%
%    \begin{column}[b]{0.4\linewidth}
%    \centering
%    \textbf{Localize}
%    \begin{equation}
%	\nonumber
%        U = L\tilde{S}^{-1/2}
%    \end{equation}
%    \end{column}
%
%    \begin{column}[b]{0.3\linewidth}
%    \centering
%    \textbf{Orthonormalize}
%    \begin{equation}
%	\nonumber
%        U = \tilde{S}^{-1/2}
%    \end{equation}
%    \end{column}
%    \end{columns}
%
%    \vspace{10mm}
%
%    \centering
%    \textbf{Orbital rotation}
%    \begin{equation}
%	\nonumber
%	\phi_i = \sum_j U_{ij} \tilde{\phi}_j \qquad \qquad F = U\tilde{F}U^{-1}
%    \end{equation}
%\end{frame}

\begin{frame}
    \frametitle{Many-electron algorithm I}
    \centering
    \textbf{Setup Fock operator} $\hat{F}^n$

    \vspace{5mm}

    \textbf{Compute Fock matrix} $F_{ij}^n = \langle\phi_i^n|\hat{F}^n|\phi_j^n\rangle$

    \vspace{5mm}

    \textbf{Diagonalize/Localize}

    \vspace{5mm}

    \textbf{Iterate BSH operators with} $\lambda_i^n \approx F_{ii}^n$
    \begin{equation}
	\nonumber
        \tilde{\phi}_i^{n+1} = -2\hat{G}_i^n \bigg[\hat{V}^n\phi_i^n -
        \sum_j\big(F_{ji}^n - \Lambda_{ji}^n\big)\phi_j^n\bigg]
    \end{equation}

    \vspace{2mm}

    \textbf{Orthonormalize} $S^{-1/2}$

    \vspace{5mm}

    \textbf{Compute KAIN update} $\delta\phi_i^n$

    \vspace{5mm}

    \textbf{Orthonormalize} $S^{-1/2}$

\end{frame}

\begin{frame}
    \frametitle{Many-electron algorithm I}
    \centering
    \textbf{Overall accuracy kept at} $\epsilon = 10^{-6}$
    \begin{center}
	\includegraphics[scale=1.0, clip, viewport = 50 550 300 740]{figures/accuracy.pdf}
    \end{center}
\end{frame}

\begin{frame}
    \frametitle{Many-electron algorithm I}
    \begin{columns}
    \begin{column}[b]{0.20\linewidth}
	\vspace{5mm}
    \end{column}
    \begin{column}[b]{0.30\linewidth}
    \begin{figure}
	\centering
	\includegraphics[scale=0.2, clip, viewport = 60 450 600 720]{figures/methane.pdf}\\
	\vspace{5mm}
    \end{figure}
    \end{column}
    \begin{column}[b]{0.50\linewidth}
    \begin{figure}
	\begin{center}
	\includegraphics[scale=0.45, clip, viewport = 320 200 520 400]{figures/methaneGrid.pdf}\\
	\end{center}
    \end{figure}
    \end{column}
    \end{columns}
    \begin{center}
	\includegraphics[scale=0.6, clip, viewport = 50 350 550 540]{figures/convergence.pdf}
    \end{center}
\end{frame}

\begin{frame}
    \frametitle{Many-electron algorithm I}
    Results displaying convergence vs final accuracy (Ca)
\end{frame}

\begin{frame}
    \frametitle{Fock matrix update}
    \centering
    \textbf{Using the relation}
    \begin{equation}
        \nonumber
        -2\hat{G}_i^n = \big(\hat{T} - \lambda_i^n\big)^{-1}
    \end{equation}

    \vspace{3mm}

    \textbf{Manipulating the Fock matrix expression}
    \begin{align}
        \nonumber
        \tilde{F}_{ij}^{n+1}    &= 
        \langle\tilde{\phi}_i^{n+1} |
        \hat{T} + \hat{V}^{n+1}     |
        \tilde{\phi}_j^{n+1}\rangle\\
        \nonumber
			    &= 
        \langle\tilde{\phi}_i^{n+1} |
        \hat{T} - \lambda_j^n       |
        \tilde{\phi}_j^{n+1}\rangle +
        \langle\tilde{\phi}_i^{n+1} |
        \hat{V}^{n+1} + \lambda_j^n |
        \tilde{\phi}_j^{n+1}\rangle\\
        \nonumber
        &\ \ \vdots
    \end{align}

    \textbf{Potential updates}
    \begin{equation}
        \nonumber
        \big(\Delta\tilde{F}_{pot}^n\big)_{ij} =
        \langle\tilde{\phi}_i^{n+1} |
        \hat{V}^n                       |
        \Delta\tilde{\phi}_j^n\rangle + 
        \langle\tilde{\phi}_i^{n+1} |
        \Delta\hat{V}^n                 |
        \phi_j^n\rangle
    \end{equation}

    \vspace{3mm}

    \textbf{Overlap updates}
    \begin{equation}
    \nonumber
        \big(\Delta\tilde{S}_1^n\big)_{ij} =
        \langle\Delta\tilde{\phi}_i^n | \phi_j^n\rangle \qquad \qquad
        \big(\Delta \tilde{S}_2^n\big)_{ij} =
        \langle\tilde{\phi}_i^{n+1} | \Delta\tilde{\phi}_j^n\rangle
    \end{equation}

    \vspace{3mm}

    \centering
    \textbf{Fock matrix without kinetic operator}
    \begin{equation}
        \nonumber
        \tilde{F}^{n+1} = F^{n} + 
        \Delta \tilde{S}_1^n F^n +
        \Delta \tilde{S}_2^n \Lambda^n +
        \Delta \tilde{F}_{pot}^n
    \end{equation}
\end{frame}

\begin{frame}
    \frametitle{Many-electron algorithm II}
    \centering
    \textbf{Setup Fock operator} $\hat{F}^n$

    \vspace{8mm}

    \textbf{Iterate BSH operators with} $\lambda_i^n \approx F_{ii}^n$
    \begin{equation}
	\nonumber
        \tilde{\phi}_i^{n+1} = -2\hat{G}_i^n \bigg[\hat{V}^n\phi_i^n -
        \sum_j\big(F_{ji}^n - \Lambda_{ji}^n\big)\phi_j^n\bigg]
    \end{equation}

    \vspace{2mm}

    \textbf{Compute potential updates} $\Delta\hat{V}^n$

    \vspace{8mm}

    \textbf{Update Fock matrix}
    \begin{equation}
        \nonumber
        \tilde{F}^{n+1} = F^{n} + 
        \Delta \tilde{S}_1^n F^n +
        \Delta \tilde{S}_2^n \Lambda^n +
        \Delta \tilde{F}_{pot}^n
    \end{equation}

    \vspace{5mm}

    \textbf{Orthonormalize} $\tilde{S}^{-1/2}$

\end{frame}

\begin{frame}
    \frametitle{Many-electron algorithm II}
    Results displaying convergence vs final accuracy (Ca)
\end{frame}

%\begin{frame}
%    \frametitle{Accurate calculations}
%    \centering
%    LDA energies in atomic units (Hartree)
%    \begin{table}
%	\tiny
%	\centering
%        \begin{tabular}{lr@{.}lr@{.}lr@{.}lr@{.}lr@{.}lr@{.}l}
%	    \hline
%	    \hline
%	    &
%	    \multicolumn{4}{c}{Helium}&\multicolumn{4}{c}{Neon}&\multicolumn{4}{c}{Argon}\\
%	    &
%	    \multicolumn{2}{c}{HOMO}&\multicolumn{2}{c}{Total}&
%	    \multicolumn{2}{c}{HOMO}&\multicolumn{2}{c}{Total}&
%	    \multicolumn{2}{c}{HOMO}&\multicolumn{2}{c}{Total}\\
%	    \hline
%	    &\multicolumn{4}{c}{}&\multicolumn{4}{c}{}&\multicolumn{4}{c}{}\\
%	    MRChem $\epsilon=10^{-3}$&	-0&570467&-2&8348568&-0&496833&-128&262186&-0&387692&-525&966790\\
%	    MRChem $\epsilon=10^{-5}$&	-0&570424&-2&8348352&-0&498035&-128&233472&-0&382348&-525&946109\\
%	    MRChem $\epsilon=10^{-7}$&	-0&570425&-2&8348836&-0&498034&-128&233481&-0&382330&-525&946196\\
%	    &\multicolumn{4}{c}{}&\multicolumn{4}{c}{}&\multicolumn{4}{c}{}\\
%	    NIST&			-0&570425&-2&8348836&-0&498034&-128&233481&-0&382330&-525&946195\\
%	    &\multicolumn{4}{c}{}&\multicolumn{4}{c}{}&\multicolumn{4}{c}{}\\
%	    aug-cc-pV6Z&		-0&570424&-2&8348289&-0&498027&-128&233402&-0&382323&-525&944181\\
%	    aug-cc-pV5Z&		-0&570417&-2&8347859&-0&498059&-128&232889&-0&382388&-525&942021\\
%	    aug-cc-pVQZ&		-0&570406&-2&8346891&-0&498302&-128&229212&-0&382463&-525&938021\\
%	    aug-cc-pVTZ&		-0&570260&-2&8343489&-0&498859&-128&218459&-0&382838&-525&933682\\
%	    aug-cc-pVDZ&		-0&569386&-2&8291516&-0&498201&-128&176831&-0&382143&-525&915702\\
%	    &\multicolumn{4}{c}{}&\multicolumn{4}{c}{}&\multicolumn{4}{c}{}\\
%	    \hline
%	    \hline
%	\end{tabular}
%    \end{table}
%    \it{NIST: National Institute of Standards and Technology (Basis set limit)}\\
%\ \\
%\ \\
%\ \\
%\begin{itemize}
%    \item We are able to attain \textbf{considerably higher} accuracy than high-quality Gaussian basis sets
%    \item Energies are not variational, but \textbf{basis set limit} within the requested precision
%    \item Calculations are still more expensive than conventional methods
%\end{itemize}
%\end{frame}

%\begin{frame}
%    \frametitle{Accurate calculations}
%    \centering
%    Replacing the exchange-correlation potential $v_{xc}$ with the exact exchange operator
%    \begin{equation}
%	\nonumber
%	\hat{K}\phi_i(\boldsymbol{r}) = \sum_j \phi_j(\boldsymbol{r}) \int P(\boldsymbol{r}-\boldsymbol{r}')
%	    \left[\phi_i(\boldsymbol{r}')\phi_j(\boldsymbol{r}')\right] d\boldsymbol{r}'
%    \end{equation}
%    gives the Hartree-Fock equations, which can be solved by the same iterative methods
%    \ \\
%    \ \\
%\begin{table}
%\tiny
%\begin{tabular}{cllll}
%\hline   
%\hline
%\multicolumn{5}{c}{Total Hartree-Fock energies in atomic units (Hartree)}\\
%&\multicolumn{1}{c}{H$_2$O}
%&\multicolumn{1}{c}{H$_2$O$_2$}
%&\multicolumn{1}{c}{CO}
%&\multicolumn{1}{c}{CO$_2$}\\
%\hline 
%            		    &               &               &               &               \\
%MRChem $\epsilon=10^{-5}$   & -76.067611455 & -150.85253297 & -112.79087294 & -187.72538886 \\
%MRChem $\epsilon=10^{-6}$   & -76.067556696 & -150.85249254 & -112.79069389 & -187.72541991 \\
%MRChem $\epsilon=10^{-7}$   & -76.067535613 & -150.85246986 & -112.79081263 & -187.72538522 \\
%MRChem $\epsilon=10^{-8}$   & -76.067535431 & -150.85247037 & -112.79081269 & -187.72538560 \\
%            		    &               &               &               &               \\
%Est. HF limit		    & -76.0675      & -150.8525     & -112.7908     & -187.7254     \\
%            		    &               &               &               &               \\
%aug-cc-pCV5Z		    & -76.067379371 & -150.85218780 & -112.79063514 & -187.72508317 \\
%aug-cc-pCVQZ		    & -76.066140457 & -150.84985235 & -112.78919290 & -187.72260431 \\
%            		    &               &               &               &               \\
%\hline   
%\hline   
%\end{tabular}
%\end{table}
%\ \\
%\ \\
%\ \\
%\begin{itemize}
%    \item We are able to attain \textbf{considerably higher} accuracy than high-quality Gaussian basis sets
%    \item Energies are not variational, but \textbf{basis set limit} within the requested precision
%    \item Calculations are still more expensive than conventional methods
%\end{itemize}
%\end{frame}

